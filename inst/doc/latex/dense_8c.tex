\hypertarget{dense_8c}{
\subsection{dense.c File Reference}
\label{dense_8c}\index{dense.c@{dense.c}}
}
{\tt \#include \char`\"{}dense.h\char`\"{}}\par
{\tt \#include \char`\"{}Mutils.h\char`\"{}}\par
\subsubsection*{Functions}
\begin{CompactItemize}
\item 
static int \hyperlink{dense_8c_a0}{left\_\-cyclic} (double x\mbox{[}$\,$\mbox{]}, int ldx, int j, int k, double cosines\mbox{[}$\,$\mbox{]}, double sines\mbox{[}$\,$\mbox{]})
\item 
static SEXP \hyperlink{dense_8c_a1}{get\-Givens} (double x\mbox{[}$\,$\mbox{]}, int ldx, int jmin, int rank)
\item 
SEXP \hyperlink{dense_8c_a2}{check\-Givens} (SEXP X, SEXP jmin, SEXP rank)
\item 
SEXP \hyperlink{dense_8c_a3}{lsq\_\-dense\_\-Chol} (SEXP X, SEXP y)
\item 
SEXP \hyperlink{dense_8c_a4}{lsq\_\-dense\_\-QR} (SEXP X, SEXP y)
\item 
SEXP \hyperlink{dense_8c_a5}{lapack\_\-qr} (SEXP Xin, SEXP tl)
\end{CompactItemize}


\subsubsection{Function Documentation}
\hypertarget{dense_8c_a2}{
\index{dense.c@{dense.c}!checkGivens@{checkGivens}}
\index{checkGivens@{checkGivens}!dense.c@{dense.c}}
\paragraph[checkGivens]{\setlength{\rightskip}{0pt plus 5cm}SEXP check\-Givens (SEXP {\em X}, SEXP {\em jmin}, SEXP {\em rank})}\hfill}
\label{dense_8c_a2}


\hypertarget{dense_8c_a1}{
\index{dense.c@{dense.c}!getGivens@{getGivens}}
\index{getGivens@{getGivens}!dense.c@{dense.c}}
\paragraph[getGivens]{\setlength{\rightskip}{0pt plus 5cm}static SEXP get\-Givens (double {\em x}\mbox{[}$\,$\mbox{]}, int {\em ldx}, int {\em jmin}, int {\em rank})\hspace{0.3cm}{\tt  \mbox{[}static\mbox{]}}}\hfill}
\label{dense_8c_a1}


\hypertarget{dense_8c_a5}{
\index{dense.c@{dense.c}!lapack_qr@{lapack\_\-qr}}
\index{lapack_qr@{lapack\_\-qr}!dense.c@{dense.c}}
\paragraph[lapack\_\-qr]{\setlength{\rightskip}{0pt plus 5cm}SEXP lapack\_\-qr (SEXP {\em Xin}, SEXP {\em tl})}\hfill}
\label{dense_8c_a5}


\hypertarget{dense_8c_a0}{
\index{dense.c@{dense.c}!left_cyclic@{left\_\-cyclic}}
\index{left_cyclic@{left\_\-cyclic}!dense.c@{dense.c}}
\paragraph[left\_\-cyclic]{\setlength{\rightskip}{0pt plus 5cm}static int left\_\-cyclic (double {\em x}\mbox{[}$\,$\mbox{]}, int {\em ldx}, int {\em j}, int {\em k}, double {\em cosines}\mbox{[}$\,$\mbox{]}, double {\em sines}\mbox{[}$\,$\mbox{]})\hspace{0.3cm}{\tt  \mbox{[}static\mbox{]}}}\hfill}
\label{dense_8c_a0}


Perform a left cyclic shift of columns j to k in the upper triangular matrix x, then restore it to upper triangular form with Givens rotations. The algorithm is based on the Fortran routine DCHEX from Linpack.

The lower triangle of x is not modified.

\begin{Desc}
\item[Parameters:]
\begin{description}
\item[{\em x}]Matrix stored in column-major order \item[{\em ldx}]leading dimension of x \item[{\em j}]column number (0-based) that will be shifted to position k \item[{\em k}]last column number (0-based) to be shifted \item[{\em cosines}]cosines of the Givens rotations \item[{\em sines}]sines of the Givens rotations\end{description}
\end{Desc}
\begin{Desc}
\item[Returns:]0 for success \end{Desc}
\hypertarget{dense_8c_a3}{
\index{dense.c@{dense.c}!lsq_dense_Chol@{lsq\_\-dense\_\-Chol}}
\index{lsq_dense_Chol@{lsq\_\-dense\_\-Chol}!dense.c@{dense.c}}
\paragraph[lsq\_\-dense\_\-Chol]{\setlength{\rightskip}{0pt plus 5cm}SEXP lsq\_\-dense\_\-Chol (SEXP {\em X}, SEXP {\em y})}\hfill}
\label{dense_8c_a3}


\hypertarget{dense_8c_a4}{
\index{dense.c@{dense.c}!lsq_dense_QR@{lsq\_\-dense\_\-QR}}
\index{lsq_dense_QR@{lsq\_\-dense\_\-QR}!dense.c@{dense.c}}
\paragraph[lsq\_\-dense\_\-QR]{\setlength{\rightskip}{0pt plus 5cm}SEXP lsq\_\-dense\_\-QR (SEXP {\em X}, SEXP {\em y})}\hfill}
\label{dense_8c_a4}


