\documentclass{article}
\usepackage{myVignette}
\usepackage[authoryear,round]{natbib}
\bibliographystyle{plainnat}
%%\VignetteIndexEntry{Comparisons of Least Squares calculation speeds}
%%\VignetteDepends{Matrix}
\usepackage{/usr/lib/R/share/texmf/Sweave}
\begin{document}

\setkeys{Gin}{width=\textwidth}
\title{Comparing Least Squares Calculations}
\author{Douglas Bates\\R Development Core Team\\\email{Douglas.Bates@R-project.org}}
\date{\today}
\maketitle
\begin{abstract}
  Many statistics methods require one or more least squares problems
  to be solved.  There are several ways to perform this calculation,
  using objects from the base R system and using objects in the
  classes defined in the \code{Matrix} package.
  
  We compare the speed of some of these methods on a very small
  example and on a example for which the model matrix is large and
  sparse.
\end{abstract}

\section{Linear least squares calculations}
\label{sec:LeastSquares}

Many statistical techniques require least squares solutions
\begin{equation}
  \label{eq:LeastSquares}
  \widehat{\bm{\beta}}=
  \arg\min_{\bm{\beta}}\left\|\bm{y}-\bX\bm{\beta}\right\|^2
\end{equation}
where $\bX$ is an $n\times p$ model matrix ($p\leq n$), $\bm{y}$ is
$n$-dimensional and $\bm{\beta}$ is $p$ dimensional.  Most statistics
texts state that the solution to (\ref{eq:LeastSquares}) is
\begin{equation}
  \label{eq:XPX}
  \widehat{\bm{\beta}}=\left(\bX\trans\bX\right)^{-1}\bX\trans\bm{y}
\end{equation}
when $\bX$ has full column rank (i.e. the columns of $\bX$ are
linearly independent) and all too frequently it is calculated in
exactly this way.


\subsection{A small example}
\label{sec:smallLSQ}

As an example, let's create a model matrix, \code{mm}, and corresponding
response vector, \code{y}, for a simple linear regression model using
the \code{Formaldehyde} data.
\begin{Schunk}
\begin{Sinput}
> data(Formaldehyde)
> str(Formaldehyde)
\end{Sinput}
\begin{Soutput}
`data.frame':	6 obs. of  2 variables:
 $ carb  : num  0.1 0.3 0.5 0.6 0.7 0.9
 $ optden: num  0.086 0.269 0.446 0.538 0.626 0.782
\end{Soutput}
\begin{Sinput}
> print(mm <- cbind(1, Formaldehyde$carb))
\end{Sinput}
\begin{Soutput}
     [,1] [,2]
[1,]    1  0.1
[2,]    1  0.3
[3,]    1  0.5
[4,]    1  0.6
[5,]    1  0.7
[6,]    1  0.9
\end{Soutput}
\begin{Sinput}
> print(y <- Formaldehyde$optden)
\end{Sinput}
\begin{Soutput}
[1] 0.086 0.269 0.446 0.538 0.626 0.782
\end{Soutput}
\end{Schunk}
Using \code{t} to evaluate
the transpose, \code{solve} to take an inverse, and the \code{\%*\%}
operator for matrix multiplication, we can translate \ref{eq:XPX} into
the \Slang{} as
\begin{Schunk}
\begin{Sinput}
> solve(t(mm) %*% mm) %*% t(mm) %*% y
\end{Sinput}
\begin{Soutput}
            [,1]
[1,] 0.005085714
[2,] 0.876285714
\end{Soutput}
\end{Schunk}

On modern computers this calculation is performed so quickly that it cannot
be timed accurately in \RR{}
\begin{Schunk}
\begin{Sinput}
> system.time(solve(t(mm) %*% mm) %*% t(mm) %*% y, gc = TRUE)
\end{Sinput}
\begin{Soutput}
[1] 0 0 0 0 0
\end{Soutput}
\end{Schunk}
and it provides essentially the same results as the standard
\code{lm.fit} function that is called by \code{lm}.
\begin{Schunk}
\begin{Sinput}
> dput(c(solve(t(mm) %*% mm) %*% t(mm) %*% y))
\end{Sinput}
\begin{Soutput}
c(0.00508571428571368, 0.876285714285715)
\end{Soutput}
\begin{Sinput}
> dput(lm.fit(mm, y)$coefficients)
\end{Sinput}
\begin{Soutput}
structure(c(0.00508571428571436, 0.876285714285714), .Names = c("x1", 
"x2"))
\end{Soutput}
\end{Schunk}
%$

\subsection{A large example}
\label{sec:largeLSQ}

For a large, ill-conditioned least squares problem, such as that
described in \citet{koen:ng:2003}, the literal translation of
(\ref{eq:XPX}) does not perform well.
\begin{Schunk}
\begin{Sinput}
> library(Matrix)
\end{Sinput}
\begin{Soutput}
Attaching package: 'Matrix'


	The following object(s) are masked _by_ .GlobalEnv :

	 mm y 
\end{Soutput}
\begin{Sinput}
> data(mm, package = "Matrix")
> data(y, package = "Matrix")
> mm <- as(mm, "matrix")
> dim(mm)
\end{Sinput}
\begin{Soutput}
[1] 1850  712
\end{Soutput}
\begin{Sinput}
> system.time(naive.sol <- solve(t(mm) %*% mm) %*% t(mm) %*% 
+     y, gc = TRUE)
\end{Sinput}
\begin{Soutput}
[1] 3.19 0.09 3.37 0.00 0.00
\end{Soutput}
\end{Schunk}

Because the calculation of a ``cross-product'' matrix, such as
$\bX\trans\bX$ or $\bX\trans\bm{y}$, is a common operation in
statistics, the \code{crossprod} function has been provided to do
this efficiently.  In the single argument form \code{crossprod(mm)}
calculates $\bX\trans\bX$, taking advantage of the symmetry of the
product.  That is, instead of calculating the $712^2=506944$ elements of
$\bX\trans\bX$ separately, it only calculates the $(712\cdot
713)/2=253828$ elements in the upper triangle and replicates them in
the lower triangle. Furthermore, there is no need to calculate the
inverse of a matrix explicitly when solving a
linear system of equations.  When the two argument form of the \code{solve}
function is used the linear system
\begin{equation}
  \label{eq:LSQsol}
  \left(\bX\trans\bX\right) \widehat{\bm{\beta}} = \bX\trans\by
\end{equation}
is solved directly.

Combining these optimizations we obtain
\begin{Schunk}
\begin{Sinput}
> system.time(cpod.sol <- solve(crossprod(mm), crossprod(mm, 
+     y)), gc = TRUE)
\end{Sinput}
\begin{Soutput}
[1] 0.65 0.03 0.68 0.00 0.00
\end{Soutput}
\begin{Sinput}
> all.equal(naive.sol, cpod.sol)
\end{Sinput}
\begin{Soutput}
[1] TRUE
\end{Soutput}
\end{Schunk}

On this computer (2.0 GHz Pentium-4, 1 GB Memory, Goto's BLAS) the
crossprod form of the calculation is about four times as fast as the
naive calculation.  In fact, the entire crossprod solution is
faster than simply calculating $\bX\trans\bX$ the naive way.
\begin{Schunk}
\begin{Sinput}
> system.time(t(mm) %*% mm, gc = TRUE)
\end{Sinput}
\begin{Soutput}
[1] 0.81 0.02 0.84 0.00 0.00
\end{Soutput}
\end{Schunk}

\subsection{Least squares calculations with Matrix classes}
\label{sec:MatrixLSQ}

The \code{crossprod} function applied to a single matrix takes
advantage of symmetry when calculating the product but does not retain
the information that the product is symmetric (and positive
semidefinite).  As a result the solution of (\ref{eq:LSQsol}) is
performed using general linear system solver based on an LU
decomposition when it would be faster, and more stable numerically, to
use a Cholesky decomposition.  The Cholesky decomposition could be used
but it is rather awkward
\begin{Schunk}
\begin{Sinput}
> system.time(ch <- chol(crossprod(mm)), gc = TRUE)
\end{Sinput}
\begin{Soutput}
[1] 0.47 0.02 0.49 0.00 0.00
\end{Soutput}
\begin{Sinput}
> system.time(chol.sol <- backsolve(ch, forwardsolve(ch, crossprod(mm, 
+     y), upper = TRUE, trans = TRUE)), gc = TRUE)
\end{Sinput}
\begin{Soutput}
[1] 0.12 0.03 0.15 0.00 0.00
\end{Soutput}
\begin{Sinput}
> all.equal(chol.sol, naive.sol)
\end{Sinput}
\begin{Soutput}
[1] TRUE
\end{Soutput}
\end{Schunk}

The \code{Matrix} package uses the S4 class system
\citep{R:Chambers:1998} to retain information on the structure of
matrices from the intermediate calculations.  A general matrix in
dense storage, created by the \code{Matrix} function, has class
\code{"dgeMatrix"} but its cross-product has class \code{"dpoMatrix"}.
The \code{solve} methods for the \code{"dpoMatrix"} class use the
Cholesky decomposition.
\begin{Schunk}
\begin{Sinput}
> data(mm, package = "Matrix")
> mm <- as(mm, "dgeMatrix")
> class(crossprod(mm))
\end{Sinput}
\begin{Soutput}
[1] "dpoMatrix"
attr(,"package")
[1] "Matrix"
\end{Soutput}
\begin{Sinput}
> system.time(Mat.sol <- solve(crossprod(mm), crossprod(mm, 
+     y)), gc = TRUE)
\end{Sinput}
\begin{Soutput}
[1] 0.48 0.03 0.50 0.00 0.00
\end{Soutput}
\begin{Sinput}
> all.equal(naive.sol, as(Mat.sol, "matrix"))
\end{Sinput}
\begin{Soutput}
[1] TRUE
\end{Soutput}
\end{Schunk}

Furthermore, any method that calculates a
decomposition or factorization stores the resulting factorization with
the original object so that it can be reused without recalculation.
\begin{Schunk}
\begin{Sinput}
> xpx <- crossprod(mm)
> xpy <- crossprod(mm, y)
> system.time(solve(xpx, xpy), gc = TRUE)
\end{Sinput}
\begin{Soutput}
[1] 0.09 0.02 0.11 0.00 0.00
\end{Soutput}
\begin{Sinput}
> system.time(solve(xpx, xpy), gc = TRUE)
\end{Sinput}
\begin{Soutput}
[1] 0.01 0.00 0.01 0.00 0.00
\end{Soutput}
\end{Schunk}

The model matrix \code{mm} is sparse; that is, most of the elements of
\code{mm} are zero.  The \code{Matrix} package incorporates special
methods for sparse matrices, which produce the fastest results of all.

\begin{Schunk}
\begin{Sinput}
> data(mm, package = "Matrix")
> class(mm)
\end{Sinput}
\begin{Soutput}
[1] "dgCMatrix"
attr(,"package")
[1] "Matrix"
\end{Soutput}
\begin{Sinput}
> system.time(sparse.sol <- solve(crossprod(mm), crossprod(mm, 
+     y)), gc = TRUE)
\end{Sinput}
\begin{Soutput}
[1] 0.05 0.00 0.06 0.00 0.00
\end{Soutput}
\begin{Sinput}
> all.equal(naive.sol, as(sparse.sol, "matrix"))
\end{Sinput}
\begin{Soutput}
[1] "Numeric: lengths (712, 0) differ"
\end{Soutput}
\end{Schunk}

As with other classes in the \code{Matrix} package, the
\code{sscMatrix} retains any factorization that has been calculated
although, in this case, the decomposition is so fast that it is
difficult to determine the difference in the solution times.

\begin{Schunk}
\begin{Sinput}
> xpx <- crossprod(mm)
> xpy <- crossprod(mm, y)
> system.time(solve(xpx, xpy), gc = TRUE)
\end{Sinput}
\begin{Soutput}
[1] 0 0 0 0 0
\end{Soutput}
\begin{Sinput}
> system.time(solve(xpx, xpy), gc = TRUE)
\end{Sinput}
\begin{Soutput}
[1] 0 0 0 0 0
\end{Soutput}
\end{Schunk}

\bibliography{Matrix}
\end{document}
