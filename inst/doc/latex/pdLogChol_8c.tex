\hypertarget{pdLogChol_8c}{
\subsection{pd\-Log\-Chol.c File Reference}
\label{pdLogChol_8c}\index{pdLogChol.c@{pdLogChol.c}}
}
{\tt \#include \char`\"{}Mutils.h\char`\"{}}\par
{\tt \#include $<$R\_\-ext/Lapack.h$>$}\par
\subsubsection*{Functions}
\begin{CompactItemize}
\item 
double \hyperlink{pdLogChol_8c_a0}{ld\_\-factor\_\-from\_\-par} (const double $\ast$par, double $\ast$factor, int nc)
\item 
double $\ast$ \hyperlink{pdLogChol_8c_a1}{gradient} (const int nc, const double $\ast$factor, const double $\ast$pars, double $\ast$value)
\item 
SEXP \hyperlink{pdLogChol_8c_a2}{pd\-Log\-Chol\_\-LMEhessian} (SEXP x, SEXP Ain, SEXP Hin, SEXP nlev)
\item 
SEXP \hyperlink{pdLogChol_8c_a3}{pd\-Log\-Chol\_\-LMEgradient} (SEXP x, SEXP Ain, SEXP nlev)
\item 
SEXP \hyperlink{pdLogChol_8c_a4}{pd\-Log\-Chol\_\-pdgradient} (SEXP x)
\item 
SEXP \hyperlink{pdLogChol_8c_a5}{pd\-Log\-Chol\_\-EMupdate} (SEXP x, SEXP nlev, SEXP Ain)
\item 
SEXP \hyperlink{pdLogChol_8c_a6}{pd\-Log\-Chol\_\-coef\-Gets} (SEXP x, SEXP value)
\end{CompactItemize}


\subsubsection{Function Documentation}
\hypertarget{pdLogChol_8c_a1}{
\index{pdLogChol.c@{pd\-Log\-Chol.c}!gradient@{gradient}}
\index{gradient@{gradient}!pdLogChol.c@{pd\-Log\-Chol.c}}
\paragraph[gradient]{\setlength{\rightskip}{0pt plus 5cm}double$\ast$ gradient (const int {\em nc}, const double $\ast$ {\em factor}, const double $\ast$ {\em pars}, double $\ast$ {\em value})\hspace{0.3cm}{\tt  \mbox{[}static\mbox{]}}}\hfill}
\label{pdLogChol_8c_a1}


An internal function that calculates the gradient of the positive-definite matrix with respect to the parameters. This function is used in both pd\-Log\-Chol\_\-LMEgradient and pd\-Log\-Chol\_\-pdgradient.

\begin{Desc}
\item[Parameters:]
\begin{description}
\item[{\em nc}]number of columns (and rows) in the matrix \item[{\em pars}]parameter vector of length nc$\ast$(nc+1)/2 \item[{\em value}]array into which the results are written\end{description}
\end{Desc}
\begin{Desc}
\item[Returns:]the gradient in value \end{Desc}
\hypertarget{pdLogChol_8c_a0}{
\index{pdLogChol.c@{pd\-Log\-Chol.c}!ld_factor_from_par@{ld\_\-factor\_\-from\_\-par}}
\index{ld_factor_from_par@{ld\_\-factor\_\-from\_\-par}!pdLogChol.c@{pd\-Log\-Chol.c}}
\paragraph[ld\_\-factor\_\-from\_\-par]{\setlength{\rightskip}{0pt plus 5cm}double ld\_\-factor\_\-from\_\-par (const double $\ast$ {\em par}, double $\ast$ {\em factor}, int {\em nc})\hspace{0.3cm}{\tt  \mbox{[}static\mbox{]}}}\hfill}
\label{pdLogChol_8c_a0}


Populate the factor from the parameter vector and return the logarithm the determinant of the factor.

\begin{Desc}
\item[Parameters:]
\begin{description}
\item[{\em par}]vector of parameters \item[{\em factor}]pointer to matrix to be overwritten with the factor \item[{\em nc}]number of columns\end{description}
\end{Desc}
\begin{Desc}
\item[Returns:]logarithm of the determinant of the factor \end{Desc}
\hypertarget{pdLogChol_8c_a6}{
\index{pdLogChol.c@{pd\-Log\-Chol.c}!pdLogChol_coefGets@{pdLogChol\_\-coefGets}}
\index{pdLogChol_coefGets@{pdLogChol\_\-coefGets}!pdLogChol.c@{pd\-Log\-Chol.c}}
\paragraph[pdLogChol\_\-coefGets]{\setlength{\rightskip}{0pt plus 5cm}SEXP pd\-Log\-Chol\_\-coef\-Gets (SEXP {\em x}, SEXP {\em value})}\hfill}
\label{pdLogChol_8c_a6}


\hypertarget{pdLogChol_8c_a5}{
\index{pdLogChol.c@{pd\-Log\-Chol.c}!pdLogChol_EMupdate@{pdLogChol\_\-EMupdate}}
\index{pdLogChol_EMupdate@{pdLogChol\_\-EMupdate}!pdLogChol.c@{pd\-Log\-Chol.c}}
\paragraph[pdLogChol\_\-EMupdate]{\setlength{\rightskip}{0pt plus 5cm}SEXP pd\-Log\-Chol\_\-EMupdate (SEXP {\em x}, SEXP {\em nlev}, SEXP {\em Ain})}\hfill}
\label{pdLogChol_8c_a5}


Perform an EM update on a pd\-Log\-Chol object.

\begin{Desc}
\item[Parameters:]
\begin{description}
\item[{\em x}]Pointer to a pd\-Log\-Chol object \item[{\em nlev}]An integer object - the number of levels in the grouping factor \item[{\em Ain}]An upper triangular matrix object\end{description}
\end{Desc}
\begin{Desc}
\item[Returns:]The updated pd\-Log\-Chol object x \end{Desc}
\hypertarget{pdLogChol_8c_a3}{
\index{pdLogChol.c@{pd\-Log\-Chol.c}!pdLogChol_LMEgradient@{pdLogChol\_\-LMEgradient}}
\index{pdLogChol_LMEgradient@{pdLogChol\_\-LMEgradient}!pdLogChol.c@{pd\-Log\-Chol.c}}
\paragraph[pdLogChol\_\-LMEgradient]{\setlength{\rightskip}{0pt plus 5cm}SEXP pd\-Log\-Chol\_\-LMEgradient (SEXP {\em x}, SEXP {\em Ain}, SEXP {\em nlev})}\hfill}
\label{pdLogChol_8c_a3}


LMEgradient implementation for the pd\-Log\-Chol class

\begin{Desc}
\item[Parameters:]
\begin{description}
\item[{\em x}]Pointer to a pd\-Log\-Chol object \item[{\em Ain}]Pointer to an upper-triangular double precision square matrix \item[{\em nlev}]Pointer to an integer scalar giving the number of levels\end{description}
\end{Desc}
\begin{Desc}
\item[Returns:]Pointer to a REAL gradient vector \end{Desc}
\hypertarget{pdLogChol_8c_a2}{
\index{pdLogChol.c@{pd\-Log\-Chol.c}!pdLogChol_LMEhessian@{pdLogChol\_\-LMEhessian}}
\index{pdLogChol_LMEhessian@{pdLogChol\_\-LMEhessian}!pdLogChol.c@{pd\-Log\-Chol.c}}
\paragraph[pdLogChol\_\-LMEhessian]{\setlength{\rightskip}{0pt plus 5cm}SEXP pd\-Log\-Chol\_\-LMEhessian (SEXP {\em x}, SEXP {\em Ain}, SEXP {\em Hin}, SEXP {\em nlev})}\hfill}
\label{pdLogChol_8c_a2}


\hypertarget{pdLogChol_8c_a4}{
\index{pdLogChol.c@{pd\-Log\-Chol.c}!pdLogChol_pdgradient@{pdLogChol\_\-pdgradient}}
\index{pdLogChol_pdgradient@{pdLogChol\_\-pdgradient}!pdLogChol.c@{pd\-Log\-Chol.c}}
\paragraph[pdLogChol\_\-pdgradient]{\setlength{\rightskip}{0pt plus 5cm}SEXP pd\-Log\-Chol\_\-pdgradient (SEXP {\em x})}\hfill}
\label{pdLogChol_8c_a4}


Implementation of the pdgradient method for pd\-Log\-Chol objects.

\begin{Desc}
\item[Parameters:]
\begin{description}
\item[{\em x}]Pointer to a pd\-Log\-Chol object\end{description}
\end{Desc}
\begin{Desc}
\item[Returns:]SEXP of a three-dimensional array with the gradient of the pdgradient with respect to the parameters. \end{Desc}
